\section{Introduction}
The Secure Shell (SSH) protocol is a popular security tool commonly enabled on hosts to supply trusted users with essential capabilities like remote login, file transfer, remote command execution, and other features requiring secure access. For many hosts, the default method of authenticating users to provide this access is the exchange of a username and password from the client to the host. Although this can be made more secure by choosing "good" passwords with a minimum length and complexity that are required to be changed regularly, this method of authenticating clients continues to be a potential  major weak point in SSH implementations, as will be discussed throughout this paper.

SSH helps guarantee security by encrypting traffic to prevent eavesdropping, providing client-side host key validation to protect against host IP spoofing, employing session integrity checks to render connection hijacking ineffective---excepting denial-of-service attacks---and more~\cite{sshbook}. However, SSH alone cannot prevent all possible attacks. Since SSH operates over TCP, weaknesses in TCP and IP that allow denial-of-service attacks can be exploited to compromise legitimate SSH connections, for example, with a SYN flood or spoofed TCP reset packet~\cite{sshbook}. Another potential vulnerability for many hosts using SSH is the ubiquitous brute force attack. In these attacks---often preceded by unwelcome port scans probing for open SSH servers---an adversary attempts to gain access to a host by automating a process to guess username and password combinations, often using a dictionary attack to try more common combinations first. An attacker commonly makes many login attempts within a relatively short time window, or they may mount a stealthy distributed attack campaign that is more difficult to distinguish from the innocuous case of legitimate users rarely failing to authenticate over time~\cite{stealthy}. The tool presented in this paper focuses on the former brute force attack pattern, although potential future work could deploy the proposed reporting framework for use with existing algorithms for detecting the more subtle attack patterns.

If an IP is reachable over the Internet and the host has enabled SSH, it will assuredly be a target for large numbers of brute force attack attempts. One honeypot server that was created in 2006 to monitor brute force SSH attack attempts experienced 6899 login attempts in 22 days, observing a total of 2741 unique usernames and 3649 unique passwords guessed~\cite{symantec}. In 2013, the security company Sucuri observed a honeypot server log 15000 brute force attempts in 7 days~\cite{sucuri1}. In both cases, the most common username tried was "root" by a majority. In another experiment, Sucuri configured five IPv4 servers with the intentionally ultra-weak credentials "root"/"password" to monitor how much time would pass before becoming compromised; the first of the servers was successfully hacked by a brute force attack in 12 minutes~\cite{sucuri2}. 

Of course, if the proper security measures are taken, these attacks can be mitigated, but attempts to gain unauthorized access will not cease just because they are unsuccessful. A host can eliminate the possibility that a brute force attack will be successful by disabling username/password authentication and instead requiring public-key authentication for SSH access. To reduce successful port scans, the administrator can use a port other than the default port 22 for SSH traffic. Additionally, the SSH port can be configured to reject all traffic originating from an IP not included in a specified whitelist. 

However, even if a host is not vulnerable to SSH brute force attacks, malicious login attempts may still occur, albeit unsuccessfully. Although this may be of no direct consequence to the secured host, the malicious source IPs instigating the attacks have many targets, some of which may remain vulnerable. Secure hosts can go a step beyond protecting themselves and use their TCP/IP activity to help to identify malicious IPs and report them to organizations and communities dedicated to maintaining public blacklists. For hosts that wish to engage in these sorts of efforts, there is a need for a tool that can run on an end host, analyze IP traffic logs to identify IPs engaged in SSH brute force attacks, and---when authorized by administrators---automatically file reports on these IPs with a reputable public blacklist.

One such blacklist is maintained by the AbuseIPDB project, a public online database dedicated to combatting abusive activity on the internet~\cite{abuse}. AbuseIPDB maintains a blacklist of abusive IPs available for webmasters, sys admins, and anyone interested in IP security. The tool introduced in this paper, irpSSHa, allows users to query large sets of IP traffic flow data for potential SSH attackers and cross references these results with AbuseIPDB records to identify potentially hostile IPs. Once identified, the user can use irpSSHa's interactive command prompt to file reports for any and all of the suspicious IPs with AbuseIPDB directly.
\label{sec:intro}

